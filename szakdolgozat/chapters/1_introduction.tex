\Chapter{Bevezetés}

Az utóbbi évtizedben a szoftverfejlesztés területén radikális változások mentek végbe, melyek közül kiemelkedik a konténerizáció technológiájának elterjedése. A konténerizáció lehetőséget nyújt arra, hogy az alkalmazásokat és azok függőségeit elszigetelt környezetben futtassuk, így biztosítva a hordozhatóságot és az egységes működést különféle környezetek között. Ez a megközelítés nagyban hozzájárult a fejlesztési és üzemeltetési (DevOps) folyamatok felgyorsításához, mivel megoldotta azokat a kihívásokat, amelyek a hagyományos telepítési módszerekkel kapcsolatban felmerültek, például az inkompatibilis rendszerváltozatok és a verziókezelési problémák.

A konténerizáció célja, hogy egy alkalmazás minden szükséges függőséggel és beállítással egy úgynevezett „konténerben” legyen tárolva, amelyet bármely platformon futtathatunk, függetlenül a helyi környezettől. A konténerek egyedülállóan hordozhatóak, gyorsan elindíthatóak és leállíthatóak, és az operációs rendszer különböző elemeit is megoszthatják, ami hatékony erőforrás-felhasználást eredményez. Míg a hagyományos virtuális gépek különálló operációs rendszert igényelnek, a konténerek az alap operációs rendszert használják, így kisebb méretűek és gyorsabbak.

A Docker az egyik legismertebb és legelterjedtebb konténerizációs platform, amely egyértelmű eszközöket biztosít a konténerek létrehozásához, menedzseléséhez és telepítéséhez. A Docker lehetővé teszi a fejlesztők számára, hogy „Dockerfile”-ok segítségével meghatározzák az alkalmazás környezetét, melyet bármely fejlesztői vagy éles rendszeren pontosan ugyanúgy futtathatnak. Ez egyértelmű előnyt jelent a verziók és függőségek kezelése szempontjából, mivel a fejlesztői és a gyártási környezetek közötti különbségek minimalizálhatóak.

A Docker növekvő népszerűsége és használata ugyanakkor újabb kihívásokat is magával hozott, különösen a konténerek felügyelete és kezelése terén. Ahhoz, hogy a Docker technológia teljes mértékben kihasználható legyen, hatékony és egyszerűen kezelhető felügyeleti eszközökre van szükség. A Docker Manager projekt célja egy olyan grafikus felhasználói felület (GUI) biztosítása, amely egyszerűsíti a konténerek kezelését, támogatva mind a kezdő, mind a haladó felhasználók igényeit.

A Docker Manager fejlesztésével egy olyan platformot kívántam létrehozni, amelyen keresztül a felhasználók könnyen menedzselhetik a konténereiket, legyen szó új konténerek létrehozásáról, leállításáról, monitorozásáról vagy konfigurálásáról. Ezzel a projekttel célom egy olyan eszköz létrehozása, amely nemcsak a fejlesztési folyamatokat gyorsítja fel, hanem az üzemeltetési feladatokat is jelentősen megkönnyíti.

Docker API segítségével programozott módon hozzáférhetünk a Docker szolgáltatásaihoz, például konténerek létrehozásához, konfigurálásához és monitorozásához. A Docker Manager az API-n keresztül kommunikál a Docker daemon-nal, amely lehetővé teszi a konténerkezelési feladatok automatizálását.

A Docker Manager Python nyelven készült, és a PyQt5 keretrendszert használja a felhasználói felület létrehozásához. A PyQt5 egy hatékony és széles körben használt GUI keretrendszer, amely megkönnyíti az interaktív, platformfüggetlen alkalmazások készítését. Ezzel lehetővé válik, hogy a felhasználók intuitív módon, egy átlátható grafikus felületen kezeljék a Docker-konténereket.